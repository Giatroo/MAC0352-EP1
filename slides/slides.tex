\documentclass[10pt]{beamer}

\usepackage{packages}
\title{Exercício Programa 1}
\subtitle{Broker MQTT}
\institute{IME-USP}
\author{Lucas Paiolla Forastiere, 11221911}
\date{17 de maio de 2021}

\begin{document}
    \maketitle

    \section{Detalhes de Implementação}

    \begin{frame}[t]
      \frametitle{Detalhes de Implementação}
      \framesubtitle{Os comandos}
      \begin{itemize}
        \item O broker interpreta cada comando de acordo com o primeiro byte do
          header do pacote. Assim que lemos o pacote, vemos se ele se encontra
          entre as categorias: \texttt{CONNECT\_PACKAGE},
          \texttt{PUBLISH\_PACKAGE}, \texttt{SUBSCRIBE\_PACKAGE},
          \texttt{PINGREQ\_PACKAGE} ou \texttt{DISCONNECT\_PACKAGE}.
        \item Qualquer outro pacote é ignorado pelo broker, pois ou não pertence
          ao escopo do projeto (como QoS maiores que $0$) ou é um pacote que
          deveria chegar apenas ao cliente (como \texttt{CONNACK\_PACKAGE}).
      \end{itemize}
    \end{frame}

    \begin{frame}[t]
      \frametitle{Detalhes de Implementação}
      \framesubtitle{Suposições}
      \begin{itemize}
        \item Assumi que todos os inteiros que descrevem tamanhos (descritos
          como \texttt{Variable Byte Integer} na RFC) nunca serão maiores que
          128, ou seja, oculpam apenas um byte.
        \item Da mesma forma, existem inteiros que descrevem tamanho que possuem
          dois bytes fixos. Eu assumi que esses valores sempre são menores que
          128. Ou seja, o primeiro byte do número sempre será \texttt{0x00}.
        \item Supõe-se que o número máximo de clientes conectados é algo da
          ordem de $100.000$ (mais exatamente, os PIDs dos processos designados
          a cada cliente não podem ultrapassar $300.000$).
      \end{itemize}
    \end{frame}

    \begin{frame}[t]
      \frametitle{Detalhes de Implementação}
      \framesubtitle{Suposições}
        Outras suposições foram feitas sobre os pacotes recebidos:
        \begin{itemize}
          \item Supõe-se que o \texttt{PUBLISH\_PACKAGE} e o
            \texttt{SUBSCRIBE\_PACKAGE} possuem \texttt{properties length} igual
            a zero.
        \end{itemize}
    \end{frame}

    \begin{frame}[t]
      \frametitle{Detalhes de Implementação}
      \framesubtitle{Armazenamento dos Inscritos}
      \begin{itemize}
        \item Para armazenar os inscritos e fazer a comunição entre os processos
          filhos, utilizou-se memória compartilhada com o uso do \texttt{mmap}.
        \item Existe um array com o nome dos tópicos, um array com o nome dos
          inscritos em cada tópico, um array com quantos são os incritos em cada
          tópico e um array com qual a última mensagem enviada para cada tópico.
        \item Para armazenar os clientes inscritos em um tópico, utiliza-se o
          array \texttt{topic\_subs} de tamanho $300.000$ onde cada posição
          $j,i$ é $i$ se, e somente se, o cliente de PID $i$ se inscreveu no
          tópico $j$. Caso contrário, é $-1$.
      \end{itemize}
    \end{frame}

    \section{Testes}
    \begin{frame}[t]
      \frametitle{Testes}
      \framesubtitle{Apenas o servidor}
      \begin{itemize}
        \item Quando apenas o servidor está em operação, sem nenhum cliente
          querendo conectar-se, não temos absolutamente nenhuma ação
          acontecendo. O servidor não envia ou recebe quaisquer mensagens.
      \end{itemize}
    \end{frame}

    \begin{frame}[t]
      \frametitle{Testes}
      \framesubtitle{Servidor com um inscrito e um publicador}
      \begin{itemize}
        \item Se não há clientes inscritos no broker, então qualquer publicação
          imprimirá a mensagem  ``\texttt{Tópico não encontrado.}''.
        \item De forma análoga, se um cliente tenta publicar algo em um tópico
          que não há clientes inscritos, a mesma mensagem é impressa.
        \item Quando um cliente pede para se inscrever em um tópico, esse tópico
          é criado (já que nesses testes sempre existirá apenas um inscrito) e
          o PID do processo é salvo para depois enviarmos mensagens a ele.
          O processo que recebeu o \texttt{SUBSCRIBE\_PACKAGE} cria um
          \texttt{fork} que ficará verificando se novas mensagens chegaram ao
          tópico utilizando uma espera ocupada.
      \end{itemize}
    \end{frame}

    \begin{frame}[t]
      \frametitle{Testes}
      \framesubtitle{Servidor com um inscrito e um publicador}
      \begin{itemize}
        \item Quando um cliente publica em um tópico existente, nós alteramos a
          última mensagem enviada ao tópico e avisamos o inscrito que uma nova
          mensagem está disponível. Então ele enviará exatamente o mesmo
          \texttt{PUBLISH\_PACKAGE} recebido para o cliente inscrito e a
          mensagem será enviada.
        \item Quando o cliente inscrito em um tópico se desconecta, nós
          removemos ele do tópico. Como nesse caso de teste há apenas um
          inscrito, o tópico agora terá zero inscritos e será também deletado,
          voltando o servidor para o caso em que não haviam inscritos no broker.
      \end{itemize}
    \end{frame}

    \begin{frame}[t]
      \frametitle{Testes}
      \framesubtitle{Servidor com 100 clientes}
      \begin{itemize}
        \item De forma geral, o servidor não lida bem com tantos clientes, pois
          ele fica muito congestionado. Entretanto, caso os clientes
          publicadores não mandem muitas mensagens por segundo, o servidor
          consegue lidar melhor do que quando esse fluxo de mensagens é muito
          grande.
        \item Tudo o que foi falado para o caso com dois clientes vale. Ou seja,
          enviar uma mensagem para um tópico inexistente resulta em uma mensagem
          de erro, mas o servidor continua normalmente.
        \item Além disso, o limite máximo de tópicos é $8$, então se já existem
          esses oito tópicos e um cliente tenta se inscrever em um novo tópico,
          esse cliente recebe uma mensagem de erro e é desconectado.
      \end{itemize}
    \end{frame}

    \begin{frame}
        \centering
        {\huge Obrigado!}

        \nl

        Lucas Paiolla

    \end{frame}

\end{document}
